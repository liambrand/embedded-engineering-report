\documentclass[]{report}


% Title Page
\title{CM0605 - Embedded Systems Engineering}
\author{Liam Brand}
\date{}


\begin{document}
\maketitle

	\chapter{Real Time Scheduling}
		\subsection{Basic Scheduling Analysis}
		
		\subsection{Scheduling with Shared Resources}
	
	
	\chapter{A Distributed Real-time System}
	
	
	
	\chapter{Reliability}
		\section{Section 1}
			\subsection{Question a}
				\subsubsection{Fault Prevention}
				\subsubsection{Software Fault Tolerance}
				\subsubsection{Hardware Fault Tolerance}
				
				
				
			
			\subsection{Question b}
				\subsubsection{Potential Risks}
				The nature of the car assembly line means any single failure will cause the entire system to fail, so in terms of the network only a single connected node in the distributed system needs to experience a problem for the entire system to stop working. 
				
				Timing will also be essential within the assembly line to ensure the robotic tools perform their jobs properly, so there is a risk of performance overhead negatively impacting the system too. Networked processors that are in close proximity will communicate incredibly quickly, but processors that are further away from the network will take longer to communicate and this will need to be accounted for otherwise tools may not behave as expected and, should any issues occur, the entire system will fail as previously discussed. In a similar vain, timing will need to be considered to ensure that all operations across the network are in synchronisation with each other. Messages arriving on time is a good start, but this doesn't guarantee that all of the production line robotics are cooperating as expected.
				
				\subsubsection{Architecture Recommendation}
				
				
				%\subsubsection{Fail-Operational}
				%\subsubsection{Fail-Active}
				%\subsubsection{Fail-Safe}
				%\subsubsection{High Availability}
			
			
			\subsection{Question c}



\end{document}          
